Thesis aimed to cover, explain and experiment with new approaches to facial recognition. The research in Chapter\,\ref{chapter:research} listed and covered mathematical and engineering principles of construction of convolutional neural networks. It stumbled upon some drawbacks of such design and proposed available solutions. It discusses convolutional neural networks as well as the newest capsule network approach. This throughout review of principles behind successful identification is later transformed into a state-of-the-art market scan, providing a quick insight into publicly available solutions as well as available frameworks. Fundamental benefits and advantages of each framework and solution is listed in the Chapter\,\ref{chapter:solutions}.

The very same chapter also provides an overview of available training data sets are listed, with focus on in-the-wild image data. This review includes a short description of each, accompanied by basic metric. Large scale face databases are listed as well.

This knowledge is later leveraged to create own implementation of a capsule network. The Chapter\,\ref{chapter:implementation} covers the actual design and implementation of this solution. A part from this implementation we also demonstrated how such model can be trained and shown the results.

\section{Experiment discussion}
Overall our implementation proved itself to be successful in recognition of 75\,\% individuals when trained with 11 identities. When trained to learn 42 individuals, the achieved performance drops to  This accuracy decreases dramatically with either more identities added or less training data per identity. These effects correlate but the actual cause may be n one factor only. That would require testing against a data set greater in magnitude, while balanced in amount of images per identity.

In the case of a model trained to recognize 11 different identities, we had available 50 or more images per each individual. These images covered different settings, angles and poses. This might have proved to be a great benefit to our model. On the other hand, the fewer amount of individuals to recognise there is much less parameters to be learned in the network and therefore the error margin is smaller. And since our data are small images of $32\times32$ pixels, the capsule network struggles to successfully match a greater amount of identities.

\section{Improvements and suggestions}

A capsule network provides great power at smaller scales, however, when it is facing big problems, it demands great computational powers and resources. Hence quick prototyping and full scale training result in big differences in network configurations and therefore the observed behaviours. To achieve better results with this type of network, over more identities, it would be necessary to provide sufficient amount of input data, great GPU resources. Enhancing detail on the input data form $32\times32$ pixels to more, comes with a great cost as well, since nearly every layer's parameter count is dependent on amount of input pixels. That means the resource demand is raising again. This problem may seem easy to bypass\,--\,forced retraining of the capsule network on a new label. This is topic is even newer than capsule networks itself and so far was not solved sufficiently.

Capsule networks are a still young approach to machine learning. First announcement of this approach is dated to 2017, which didn't yet provided enough time for this technology to mature. It features new and never before tried algorithms and requires more advanced mathematics and thinking than a standard and nowadays classical convolutional neural network. This thesis covered one of the first attempts to publicize and advertise the possibilities and capabilities of a capsule neural network in the field of facial recognition. The success rate of the experiments does not reach the best of the best in the field, however the results can hopefully serve as a base for future research.
