Implementation chapter is meant to cover the actual experimentation and programming part of this thesis. Reader of this document is lead through series of experiments and examples and is taken on journey to a reliable solution of the matter. On following pages you will find and reveal complexity of this problem.

This chapter aims to show and uncover every detail the author stumbled upon when he tried to implement the solutions proposed by Hinton\,\cite{capsule}, which we elaborated in detail in Chapter \ref{chapter:research}. Moreover we will discuss the differences when this solution is compared to other, related implemetations mentinoed in the Chapter \ref{chapter:solutions}.

\section{Preparation and prerequisites}

Keras with Tensorflow backend was selected as the key framework to use for this implementation. That inherently means, we are bound to use Python as a programming language. However, selection of Python is natural and reasonable anyways, since it is the most used language in the field of machine learning, artificial inteligence experimentation. Moreover due to technical limitation and proven better performance, Anacoda Python distribution is selected as the proper backend. According to various researches\,\todo{cite} and projects, Tensorflow performance fluctuates a lot since the pre-compiled packages are not allowed to use all the capabilities of each and every specific hardware combination. Therefore projects like Thoth\footnote{http://thoth-station.ninja} were created to provide dependency mesh mapping . In our example we can be satisfied by the enhanced performance of Anaconda/Conda Tensorflow distribution (from either Anaconda or Intel channels). Moreover, running Tensorflow locally on CPU is used for quick prototyping. For more demanding executions, Google Colab\footnote{https://colab.research.google.com} is selected as Jupyter notebook execution provider. All source codes to this implementation were released under Apache 2.0 license on Git Hub\footnote{https://github.com/tumido/capsnet-face}.

\section{Keras on Tensorflow}

\section{Architecture}

\subsection{Encoder}

\subsection{Decoder}

\section{Lifecycle of a model}

\subsection{Train}

\subsection{Test}

\subsection{Predict}
