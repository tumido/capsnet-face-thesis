It's essential to any living species to recognise others in their community. Evolution has allowed animals to develop many methods, how to identify fellow members. Different species use various senses to do so. Smell, vision or hearing are the most common ones. However, humankind relies mostly on their vision. We didn't develop strong perception of smell or other senses, that we can utilise on purpose or over longer distances. Therefore we mostly use our vision.

Moreover our species developed a habit to cover ourselves in clothes, which left only certain parts of our body exposed. Usually that's head and hands. And since receptors for most of our senses are placed on head, we're required to have it exposed to be able to orient in surroundings. This makes the frontal part of our head, a face, a great candidate for identifying others. We can recognise many nuances of different elements on the face. Shape of eyes, nose or mouth, colour of our eyes... That's just a subset of all the available features we use.

In this era of computers and information technologies, mankind tend to use what we know and offload our skills and capabilities to artificial intelligence. We hope to sharpen and enhance our senses; obtain more knowledge or widen our possibilities. As an example, which is more deeply elaborated in this thesis, we're trying to "teach" machines to recognise faces in the same way as we do. And there are various reasons to do so.

\section{Motivation}
Facial recognition in computer science is used in many different ways. Some use cases require people to be recognised for personal or company security purposes, for surveillance or just for our convenience.

Automating security countermeasures is a huge driving factor. It allows us to devote our time to other activities then watching over our belongings. It ensures our privacy and well being of your community. Let's list some basic use cases:

\begin{itemize}
    \item \textbf{Personal security}: unlocking a smartphone or laptop, etc.
    \item \textbf{Enterprise security and state defence}: access control, customs and border control, etc.
    \item \textbf{Surveillance}: Outlaws identification in public places, riot controls, etc.
\end{itemize}

Furthermore, automated recognition of people's faces and identification of them can facilitate us many other processes for our convenience, usually just to save us time and provide a better service to us. These use cases spans from face and smile detection on photographs to automated tagging of individual's friend on social networks.

\section{Justification}

Similarly to every field in computer science, there is an ongoing race to provide better service in facial recognition. There are multiple aspects used as a metric to define better solution and this varies for each use case. Once, it's essential to provide the best performing software for automated recognition in places where limited computational power is available. Other time, it's rather about precision and time required to recognise a face which defines success. Outrunning other solutions means better service and inventing new technologies along the way unlocks better understanding in our own thinking process. In this thesis the existing solutions will be elaborated in detail as well as principles used to achieve such results.

There are also different approaches used for creating different solutions. Sometimes a naive approach is used, other times the implementation leverages different aspects of Neural Networks. 

The aim of this thesis is to construct a very own solution to this problem, relying strictly on Neural Networks. Each part of the problem is explained and implemented. Comparison with existing solution is offered as a part of the discussion chapter~\ref{chapter:discussion}, later in this paper.

\section{Computer science approach}

\todo{divide into sub-problems}
